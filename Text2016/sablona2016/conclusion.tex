%=========================================================================
% Conclusion of Marek Kidon's thesis

V této práci jsme viděli, jak je možné navrhovat hašovací funkce evolučně
a pak zejména co musíme splnit, aby navržené hašovací funke byly skutečně
kvalitní. Viděli jsme co je jsou důležité faktory při návrhu hašovací 
funkce, co musí být splnění aby hašovací funkce pracovala dobře a jakým
okolnostem se máme naopak vyhnout.

V kapitole 2 jsme se zabývali hašováním. Uvedli jsme nezbytné pojmy 
a pochopili jsme okolnosti, za kterých hašovací tabulka funguje skutečně
dobře. Vysvětlili jsme si \textit{faktor zatížení} a zdůraznili, jak
moc důležité je, aby hašovací funkce uniformě distribuovala své výstupy.
Součástí druhé kapitoly byl také příklad, který demonstroval, jaký dopad
má nevhodně zvolená doména klíčů na výkon hašovací funkce, která pro 
vhodně zvolenoudoménu funguje bezchybně. Dále jsme si řekli, jaké jsou
nejčastější kritéria hodnocení kvality hašovacích funkcí a jak uvedli
jsme několik obecných metod, jak se hašovací funkce dá navrhnout.

Kapitola 3 nás ve svém začátku přivedla k myšlence počítání podle přírody
a k obecným faktům, které u tohoto přístupu platí. Dále se v této kapitole
zabýváme evolučnímy algoritmy, což je specifická oblast počítání podle
přírody, která je pro nás důležitá zejména tím, že evolučnímy algoritmy
budeme hledat naše hašovací funkce. Jsou zde probrané všechny důležité 
aspekty evolučních algoritmů a jsou zde také uvedení dva zástupci této
kategorie.

V poslední části se zabýváme do hloubky genetickým programováním, protže
to je náš evoluční algoritmus, který použíjeme pro hledání.

Poslední kapitola 4 je věnována návrhu řešení. Je zde prodiskutována 
reprezentace, proč je vhodné genetické programování a dále vhodné množiny
terminálů a funkcí. V neposlední řadě se zde bavíme o selekčnich mechanismech a
inicializaci nulté populace.

Asi nejdůležitější částí této kapitoly je volba fitness funkce, tedy 
prostředku ohodnocení kandidátních řešení. Jsou zde rozebrány přístupy
, které se objevili v předešlých publikacích, rozdíly oproti našemu problému
a taky odůvodnění proč je naše řešení správné a co přináší nového.
%=========================================================================
