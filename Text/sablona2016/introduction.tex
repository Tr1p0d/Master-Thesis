%=========================================================================
% Introduction of Marek Kidon's thesis

\externaldocument{obsah}

Hašovací 
funkce jsou již dobře zavedenou a neodmyslitelnou součástí moderního
počítačového světa. Jejich uplatnění je široké, nicméně nejčastěji se s
nimi setkáváme v kryptografii a hašovacích tabulkách. Hašovací
tabulky jsou významné a hojně využívané vyhledávací datové struktury, 
které při vhodném zvolení hašovací funkce pracují velmi rychle a 
efektivně. Existuje nespočet \textit{state-of-the-art} hašovacích funkcí a
jejich řady se neustále rozšiřují.
Tradiční návrh vhodné hašovací funkce pro zadanou aplikační 
doménu není triviální úkol. Pomoci si však můžeme i méně tradičními 
metodami návrhu. Jednou z nich je i návrh založený na evolčních 
algoritmech neboli evoluční návrh. Inspirací pro evoluční návrh je jev 
darwinistické reprodukce druhů. Evoluční algoritmy obecně pracují nad populací 
kandidátních řešení, kde každé kandidátní řešení je právě jeden jedinec. 
Pro vytvoření nové populace jedinců se používají 
biologií inspirované operátory. Myšlenka spočívá v postupném prohledávání 
prostoru možných řešení dokud nenalezneme takové, které svými vlastnosti 
bude převyšovat námi stanovený práh, kde jeden krok iteračního 
prohledávacího algoritmu je právě jedna generace. Cílem práce je evolučne navrhnout
hašovací funkce pro specifickou doménu \textit{IP} adres, jednoznačné identifikátory
síťového zařízení v sítích řízených internetovým protokolem. Navržené hašovací
funkce by měli předčit \textit{state-of-the-art} hašovací funkce navržené člověkem
co se do rychlosti a odolnosti vůči kolizím týče.

V kapitole 2 se budeme blíže zabývat hašováním. Neprve si zavedeme 
nezbytné pojmy a v návaznosti si popíšeme, jak fungují hašovací tabulky a 
jak je důležité vhodně zvolit hašovací funkci, popřípadě jaký dopad na 
výkonnost hašovací tabulky má nevhodně zvolená hašovací funkce.
Kapitola 3 bude zaměřena na evoluční návrh. Popíšeme si, jakou roli hraje
evoluční návrh v kontextu počítání podle přírody (natural computing).
Následující kapitola 4 bude věnována návrhu řešení. Uvedeme 
zvolenou metodu evolučního návrhu a popíšeme si její parametry. Součástí 
bude také popis fitness funkce, tedy to co budeme považovat za důležité a 
podle čeho se budeme rozhodovat při hledání našeho požadovaného řešení.
%=========================================================================
