%=========================================================================
% Conclusion of Marek Kidon's thesis

V této práci jsme viděli, jak je možné navrhovat hašovací funkce evolučně
a pak zejména co musíme splnit, aby navržené hašovací funke byly skutečně
kvalitní. Viděli jsme co jsou důležité faktory při návrhu hašovací 
funkce, co musí být splnění aby hašovací funkce pracovala dobře a jakým
okolnostem se máme naopak vyhnout.

Navrhnout dobrou hašovací funkci není jednoduchý úkol. Neexistují pevně dané normy,
standardy ani návody. Autor hašovací funkce se musí spolehnout na svoje schopnosti,
znalosti nebo intuici.

Právě zde se otevírá prostor pro nekonvenční přístup jakým k řešení
problémů přistupuje evoluční návrh. Evoluční návrh přistupuje k řešení problému
modelováním dané problematiky prohledávacím prostorem, ve kterém je možné najít
kvalitní řešení. Evoluční návrh se inspiruje procesy v přírodě konkřétně Darwinistickou
evoluční teorií.

V této práci jsme navrhovali hašovací funkce pro doménu IP adres. Navrhnutí hašovací funkce 
``na míru'' konkrétní doméně hodnot může vyústit v lepší vlastnosti hašovacích tabulek, které
jsou na ních postavené, než v případě použití konvenční, \textit{state-of-the-art}, obecných hašovacích
funkcí vytvořených člověkem.

Byly navrženy tři metody pro řešení dané problematiky. Přímý evoluční návrh, evoluční optimalizace upraveného
Merkle-Damg\r{a}rdova schémata a evoluční návrh hašovacích funkcí na míru kukaččímu hašování.
Jejich výsledky ukazují, že evoluční návrh
má v této problematice uplatnění, protože navržené hašovací funkce vykazují lepši výsledky než jejich
obecné protějšky co se do rychlosti a odolnosti vůči kolizím týče. 

Předběžné výsledky této práce byly publikovány na konference Excel@FIT 2016. Příspěvek byl vybrán
odbornou komísí k ústní prezentaci. Práce získala ocenění odborného panelu za nekonvenční přístup k návrhu hašovacích
funkcí pomocí evolučního algoritmu a ocenění odbornou věřejností za excelentní práci a její prezentaci během přehlídky. Článek
a poster si je možné prohlédnout v příloze.

V této práci byla provedena celá řada experimentů a díky rozsáhlosti dané problematiky je zde prostor pro další zkoumání
a provedení dalších experimentů. Předmětem zkoumání může být  vliv parametrů evolučního algoritmu na výsledné hašovací funkce.
Zejména zajímavé by bylo zkoumat jaký vliv na výsledky májí jiné druhy selekce a genetických operátorů a jejich podrobné nastavení.
Velmi užitečné
by bylo provést experimenty s Merkle-Damg\r{a}rdovým schématem v původní podobě. Ty by se daly porovnat s výsledky
pozměněného Merkle-Damg\r{a}rdova schémata a zároveň ověřit výsledky dosažené stejnou metodou v již existujících
pracích podobného charakteru. Neméně zajímave by bylo experimentovat s jinými schématy pro výstavbu hašovacích funkcí
jako je Merkleho strom a podobně. Kukaččí hašování se v součanosti těší velké oblibě a zařazení experimentů s jinými kvalitními
schématy pro řešení hašovacích kolizí by umožnilo tato schémata porovnávat v kontextu problematiky návrhu specifických hašovacích
funkcí.
%=========================================================================
