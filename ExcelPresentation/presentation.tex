\documentclass[11pt]{beamer}
\usetheme{default}

\usepackage[utf8]{inputenc}
\usepackage[T1]{fontenc}
\usepackage{amsmath}
\usepackage{amsfonts}
\usepackage{amssymb}
\usepackage{graphicx}

\author{Marek Kidoň}
\title{Evolutionary design of domain specific non-cryptographic hash functions}

\subtitle{}

\logo{}

\institute{Brno University of Technology}

\date{\today}

\subject{}

\setbeamercovered{transparent}

\setbeamertemplate{navigation symbols}{}

\setbeamerfont{page number in head/foot}{size=\large}
\setbeamertemplate{footline}[frame number]

\begin{document}
	\begin{frame}[plain]
		\titlepage
		 \addtocounter{framenumber}{-1}
	\end{frame}
	\begin{frame}
		\frametitle{Hash function [1/2]}
		\begin{block}{Introduction} 
		\begin{itemize}[<+->]
			\item Function of form $f_{hash} : U \to S$ for virtually any  key universe $U$ and $S = \{ 0,1,2 \ldots n \}$.
			\item $|U| \gg |S|$.
			\item One of the most important applications are hash tables and cryptography.
			\item State of the art implementations: \textit{lookup3}, \textit{Murmurhash3}, \textit{CityHash}, \textit{FarmHash}
		\end{itemize}
		\end{block}
    %\addtocounter{framenumber}{-1}
	\end{frame}
	
	\begin{frame}
		\frametitle{Hash function [2/2]}
		\begin{block}{Idea} 
			\begin{itemize}[<+->]
				\item Domain specific vs. generic hash functions. \\
				\item Idea : ``Tailor suiting'' hash function to a specific domain $\rightarrowtail$ better performance. 
			\end{itemize}
		\end{block}
	\end{frame}
	
	\begin{frame}
		\frametitle{Evolution Design}
		\begin{block}{An unconventional approach}
			\begin{itemize}[<+->]
				\item Vast variety of problems can be view as state space search problem.
				\item Inspired by the process of species reproduction.
				\item It is typically implemented as an iterative algorithm (evolutionary algorithm).
			\end{itemize}
		\end{block}
		\begin{block}{Evolution algorithm}
			\begin{itemize}[<+->]
				\item Uses a \textit{population} to represent a set of feasible solutions called \textit{individual}.
				\item New generation is bred by selecting \textit{fit} individuals using a \textit{fitness function} 
					and applying \textit{genetic operators}.
			\end{itemize}  
		\end{block}
	\end{frame}
	
	\begin{frame}
		\frametitle{Problem specification}
		\begin{itemize}[<+->]
				\item The target domain are \textit{Internet Protocol} (IP) version 4 addresses.
				\item Four different IP datasets are specified.
				\item Evolutionary design a hash function for each subset such that it outperforms
					human-created solutions.
		\end{itemize}
	\end{frame}
	\begin{frame}{Solution design}
			\begin{itemize} [<+->]
				\item \textit{Genetic programming} evolutionary algorithm
				\item Individuals are represented as \textit{abstract syntax trees}.
				\item Nodes are commonly used hashing operators such as the multiplication ($*$), addtition ($+$) or rotation ($\lll$).
				\item Leafs are IP address octets ($o_0, o_1, o_2, o_3$) and randomly selected prime numbers ($\Re$).
				\item Fitness function is defined as a ratio of number of IP addresses in the given dataset to the number of IP addresses
					individual managed to hash without a collision.
			\end{itemize}
	\end{frame}
	\begin{frame}{Results}
		
	\end{frame}
	\begin{frame}{Conclusion}
		content...
	\end{frame}
	\begin{frame}{References}
		
	\end{frame}
\end{document}